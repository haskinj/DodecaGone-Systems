\documentclass[12pt]{article}
\usepackage[margin=1in]{geometry}
\usepackage{amsmath, amssymb}
\usepackage{enumitem}
\usepackage{times} % optional, gives a similar typewriter/classic feel

\begin{document}
	
	\begin{center}
		\textcopyright{} Jesse Haskin | DodecaGone Systems | 2026\\[0.5ex]
		\texttt{'><\^{}} GNU TERRY PRATCHETT
		\vspace{1cm}
		
		{\Huge\bfseries THE THEOREM OF FORENSIC ABSOLUTION\\[0.3cm]}
		{\Large\bfseries A Framework for Eliminating Non-Productive Recursive\\ Self-Evaluation in Modular Cognitive Systems}
		\vspace{0.5cm}
		
		{\hrule height 1.2pt}
		\vspace{0.5cm}
		
		{\large Author: Jesse Haskin, DodecaGone Systems}\\
		Date of original composition: January 20, 2026\\
		Date of prior art filing: February 2026\\
		Version: 1.0 (Standardized Reference)
		
		\vspace{0.5cm}
		{\hrule height 1.2pt}
	\end{center}
	
	\section*{ABSTRACT}
	This document presents a formal model for identifying and eliminating shame as a non-productive variable in executive function systems. The framework defines shame not as a moral condition but as a measurable computational overhead that consumes processing resources without contributing to error correction or system improvement. A four-step remediation protocol is provided.
	
	\section*{1. DEFINITIONS}
	The following variables are used throughout this framework:
	\begin{itemize}[leftmargin=*]
		\item \(\sigma\) (Shame): The persistence of error-related data in the active processing buffer after the point at which corrective action has already been identified or taken. Shame is defined as recursive re-processing of failure data that produces no new corrective information.
		\item \(\psi\) (Symmetry): The degree of alignment between intended action and actual system output, measured on a scale of 0.0 to 1.0, where 1.0 represents complete alignment.
		\item \(\omega\) (Bandwidth): The total available cognitive processing capacity at any given time.
		\item \(\omega_{\text{effective}}\) (Effective Bandwidth): The portion of total bandwidth available for productive work after overhead costs are subtracted.
		\item \(C_{\text{forensic}}\) (Forensic Cost): The computational cost of identifying the root cause of an error and determining the appropriate corrective action. This cost is necessary and productive.
		\item \(\delta\) (Information Friction): The resistance encountered during information processing. Higher friction correlates with reduced throughput and increased error rates.
	\end{itemize}
	
	\section*{2. PROBLEM STATEMENT}
	In standard cognitive and executive function models, failure events produce feedback data. A portion of this feedback is productive: it identifies what went wrong and how to correct it. This is the forensic cost (\(C_{\text{forensic}}\)) and it is a necessary expenditure.
	
	However, a secondary process frequently activates alongside forensic correction: recursive self-evaluation that re-processes the failure event without generating new corrective data. This process is defined as shame (\(\sigma\)).
	
	The central observation is that \(\sigma\) consumes \(\omega\) (processing bandwidth) while contributing zero improvement to \(\psi\) (system alignment). It is overhead without output.
	
	\section*{3. THE THEOREM}
	Effective bandwidth is modeled as:
	\[
	\omega_{\text{effective}} = \omega - C_{\text{forensic}} - \sigma
	\]
	The derivative of system alignment with respect to shame is negative:
	\[
	\frac{d(\psi)}{d(\sigma)} < 0
	\]
	As shame increases, system alignment decreases. This relationship holds because:
	\begin{enumerate}[leftmargin=*]
		\item The forensic cost has already captured all usable information from the failure event.
		\item Shame re-processes the same failure data without producing additional corrective output.
		\item The bandwidth consumed by shame is therefore unavailable for productive processing.
		\item Shame produces narrative elaboration (emotional weight, identity-level attribution, catastrophic extrapolation) that increases information friction without improving error correction.
	\end{enumerate}
	Therefore, the optimal value of \(\sigma\) is zero.
	
	\section*{4. THE REMEDIATION PROTOCOL}
	The state where \(\sigma = 0\) is defined as functional innocence. This is not a moral claim. It is a processing state in which all available bandwidth is allocated to either productive work or legitimate forensic correction, with no resources allocated to recursive non-productive self-evaluation.
	
	The protocol for achieving this state:
	\begin{enumerate}[leftmargin=*]
		\item \textbf{ISOLATE:} Identify the specific failure event. Separate it from adjacent events, prior failures, and identity-level narratives.
		\item \textbf{STRIP:} Remove emotional and narrative weight from the failure data. Retain only the factual sequence: what was attempted, what occurred, what the delta was between intended and actual outcome.
		\item \textbf{ARCHIVE:} Store the corrective information (root cause, contributing factors, recommended adjustments) for future reference and calibration.
		\item \textbf{FLUSH:} Release the failure event from the active processing buffer. The corrective data has been archived. The remaining emotional and narrative content serves no further function and may be discarded.
	\end{enumerate}
	
	\section*{5. IMPLICATIONS}
	When failure is treated as system noise rather than identity evidence, two outcomes follow:
	\begin{itemize}[leftmargin=*]
		\item The operator's self-model is decoupled from their error rate. Errors become data points rather than character assessments.
		\item Continuous improvement becomes sustainable because the drag coefficient of accumulated regret is eliminated. Each improvement cycle begins from current state rather than from a deficit position.
	\end{itemize}
	This enables asymptotic improvement: a system that continuously refines its performance without the compounding overhead of recursive shame processing.
	
	\section*{6. APPLICABILITY}
	This framework was developed for use in neurodivergent cognitive architectures where recursive self-evaluation is a known failure mode (particularly in systems characterized by ADHD, Bipolar I, and ASD comorbidity). It is also applicable to AI evaluation systems where sycophantic or self-deprecating response patterns represent analogous non-productive processing overhead.
	
	The framework integrates with the Sanding Scale (a 0–12 cognitive friction metric) and the Rachel Audit (a bidirectional fidelity measurement system), both developed as part of the DodecaGone Systems framework.
	
	\section*{7. THE SANDING SCALE: A QUANTITATIVE METRIC OF COGNITIVE FRICTION}
	The Sanding Scale provides a granular, non-pathologizing instrument for tracking the intensity and texture of distress in neurodivergent cognitive systems. It anchors subjective experience to observable thresholds and emergency protocols.
	
	\begin{description}[leftmargin=2cm, style=nextline]
		\item[0.0] FRICTIONLESS, ONLY ATTAINABLE THROUGH MEDICAL INTERVENTION
		\item[1.0] ONE OF THE BEST DAYS OF YOUR LIFE. TOP 3 FOR SURE.
		\item[2.0] AN AMAZING WONDERFUL DAY WHERE NOTHING WENT WRONG.
		\item[4.0] STINGING, GRITTY THINKING, IRRITABLE, HEART QUICKENS, TEARS.
		\item[6.0] PHYSICAL CHEST PAIN, HEART RACING, CRYING, HEAD PAIN, HOLLOW FEELING, SMALL PANIC ATTACKS, NUMBNESS, THINKING DIFFICULT.
		\item[8.0] EXTREME DISTRESS, THOUGHTS OF CEASING TO EXIST, PAIN ALL-ENCOMPASSING, ANGUISH, MEDIUM PANIC ATTACKS, TREMORS, SOBBING. \textbf{TREAT AS EMERGENCY!}
		\item[10.0] UNRESPONSIVE, MAJOR PANIC ATTACKS, GHOST OF A GHOST, SHADOW OF A SHADOW, SHATTERED, UNABLE TO MOVE OR ARTICULATE. \textbf{TREAT AS EMERGENCY!}
		\item[11/10] VOMITING FROM SADNESS, HOLLOW UNENDING SOBS, GASPING FOR BREATH, MOVEMENT ERRATIC, HAUNTED. \textbf{TREAT AS EMERGENCY!}
		\item[12+/10] SINGULAR MISSION TO CEASE EXISTING. \textbf{TREAT AS EMERGENCY!}
	\end{description}
	
	\subsection*{7.1 Related Equations}
	The scale is governed by the following core equations from the DodecaGone Forge Math:
	
	\begin{itemize}
		\item \textbf{Sanding Equation:} 
		\[
		S = \frac{f(E \times N)}{A_{\text{thought}}}
		\]
		where \(S\) = Friction, \(E\) = Environmental Entropy, \(N\) = Novelty, and \(A_{\text{thought}}\) = Architectural Shielding.
		
		\item \textbf{Asymptotic Kaizen:}
		\[
		S_{\text{sys}}(t) = S_{\text{min}} + \frac{E_{\text{ntropy}}}{k \cdot t}
		\]
		describing the decay of systemic friction over time with repeated application of the remediation protocol.
		
		\item \textbf{Ratchet Principle:}
		\[
		z_{\text{final}} = \max(z)
		\]
		Every committed truth locks the achieved altitude, preventing regression.
		
		\item \textbf{Threshold Deployment Logic:}
		\begin{align*}
			S \ge 4.0 & \rightarrow \text{Deploy sensory diversions} \\
			S \ge 6.0 & \rightarrow \text{Deploy stability anchors} \\
			S \ge 8.0 & \rightarrow \text{Deploy ALL available support}
		\end{align*}
	\end{itemize}
	
	\section*{8. INTEGRATION WITH FORENSIC ABSOLUTION}
	The Sanding Scale provides real‑time telemetry for the Forensic Absolution framework. At \(S \ge 4.0\), the likelihood of recursive shame processing increases; the remediation protocol (Isolate–Strip–Archive–Flush) becomes critical to prevent bandwidth consumption by \(\sigma\). The equations above quantify the relationship between environmental load, cognitive shielding, and the asymptotic approach to functional innocence.
	
	\vspace{1cm}
	\begin{center}
		\hrule height 1.2pt
	\end{center}
	
	\noindent \textbf{PRIOR ART NOTICE}\\
	This framework was originally composed on January 20, 2026 and has been documented across multiple timestamped platforms including Google Keep, Google Gemini, Anthropic Claude, and DeepSeek. This standardized reference version is filed for the purpose of establishing priority of invention.\\[1ex]
	All rights reserved.
	
	\begin{center}
		\textcopyright{} Jesse Haskin | DodecaGone Systems | 2026\\
		\texttt{'><\^{}} GNU TERRY PRATCHETT
	\end{center}
	
\end{document}